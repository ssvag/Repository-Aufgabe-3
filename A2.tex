\documentclass{article}
\usepackage{amsmath}

\begin{document}

\centering \huge CS102 \LaTeX \ {\"U}bung
\\ \
\\ \huge Simon Svab \large
\\ \
\\ \LARGE 04. November 2014 \large
\\ \
\\ \Large \flushleft \textbf{Abschnitt 1}
\\ \
\\ \large Der Text steht hier.
\\ \
\\ \Large \textbf{Abschnitt 2: Tabelle}
\\ \ \centering
\\ \begin{tabular}{c | c | c | c} 
  \ & Punkte erhalten & Punkte m{\"o}glich & \% \\
  \hline
  Aufgabe 1 & 2 & 4 & 0.5 \\
  Aufgabe 2 & 3 & 3 & 1 \\
  Aufgabe 3 & 3 & 3 & 1 \\
 \end{tabular}
\\ \
\\ \flushleft \textbf{Abschnitt 3: Formeln}
\\ \
\\ \Large \textbf{3.1 Pythagoras} \large
\\ \
\\ \large Der Satz des Pythagoras errechnet sich wie folgt: $a^2 + b^2 = c^2$.
Daraus k{\"o}nnen wir die L{\"a}nge der Hypothenuse wie folgt berechnen: $c^2 = \sqrt {a^2+b^2}$.
\\ \
\\ \Large \textbf{3.2 Summen} \large
\\ \
\\ Wir k{\"o}nnen auch die Formel f{\"u}r eine Summe angeben: \begin{align}
s = \sum\limits_{i=1}^n i = \frac{n * (n+1)}{2}
\end{align}




\end{document}
